\section{Results on variational inequalities}
\label{sec:appendix}
The following section contains the necessary results on variational inequalities, which have been removed from the main text to improve the flow.

We start by recalling the definition of pseudo-convexity
The corresponding notion of convexity is
\begin{definition}
  Let $\Omega \subset H$ be an open subset of $H$, and let $f:\Omega \to \R$ be Gateaux-differentiable. The function $f$ is strictly pseudoconvex if
  \begin{equation}
    \ip{y-x}{(\nabla f)(x)} \geq 0 \Rightarrow f(y) > f(x)
  \end{equation}
\end{definition}
Where a strictly convex function has a strictly monotone derivative, a variant holds for strictly pseudoconvex functions. First, we need to introduce strict pseudomonotonicity.
\begin{definition}
 The operator $T: K \to H$ is strictly pseudomonotone if for every pair $x\neq y$ we have
 \begin{align}
   \ip{x-y}{Ty} \geq 0 \Rightarrow \ip{x-y}{Tx} > 0
 \end{align}
\end{definition}
Proving strict pseudomonotonicity in itself can be hard, but thankfully the notions of strict pseudomonotonicity and strict pseudoconvexity are related.
\begin{theorem}[Theorem 12.13, p. 521 \citep{hadjisavvas2006handbook}]
  Let $\Omega \subset H$ be an open convex subset, and let $f:\Omega \to \R$ be Gateaux differentiable. Then $f$ is strictly pseudoconvex if and only if $\nabla f$ is strictly pseudomonotone.
\end{theorem}
Minimizing a differentiable strictly pseudoconvex $f$ function over a convex set $K$ is equivalent to solving the variational inequality
\begin{equation}
  x \in K, ~ \ip{(\nabla f)(x)}{x-y} \geq 0, \forall y\in K
\end{equation}
The problem of existence and uniqueness of a Nash equilibrium has been reduced to a problem of existence and uniqueness of a variational inequality. Whether a variational inequality given by a pseudomonotone operator has a solution can be determined by \citep[Theorem 3.4]{maugeri2009existence}, which we abridge:
\begin{theorem}
  \label{thm:existence}
  Let $K$ be a closed convex set and $A : K \to H$  a pseudo-
  monotone map which is continuous on finite dimensional subspaces of $H$. Then the fol-
  lowing statements are equivalent:
\end{theorem}
\begin{enumerate}
  \item The variational inequality $\ip{Ax}{y-x} \geq 0$ admits solutions.
  \item There exists a point $u_0 \in K$ such that the set
  \begin{equation}
    \{v \in K : \ip{A(v)}{v-u_0} < 0\}
  \end{equation}
  is bounded.
\end{enumerate}
\begin{comment}


\citep[Theorem 3.6]{maugeri2009existence}, which we restate.
\begin{theorem}
  \label{thm:existence}
  Let $K\subset H$ be a closed convex set and $T:K\to H$ a pseudomonotone operator which is lower hemicontinuous along line segments, i.e. for all $x,y\in H$ the mapping $\xi \mapsto \ip{T\xi,}{x-y}$ is lower semicontinuous for $\xi \in \{\eta \in H : \eta = tx+(1-t)y, \quad t\in [0,1]\}$. Assume that there exists $u_0 \in K$ and $R> \norm{u_0}$ such that
  \begin{equation}
    \ip{Tv}{v-u_0} \geq 0, \forall v \in K \cap \{v \in H : \norm{v} = R \}
  \end{equation}
  then the variational inequality $VI(T,K)$ has a solution.
\end{theorem}
\end{comment}
\begin{remark}
  \label{rem:weak_compact}
  Weak compactness of $K$ also ensures that $VI(T,K)$ has a solution \citep[Theorem 12.1, P. 510]{hadjisavvas2006handbook}, giving existence of solutions in the finite-dimensional case.
\end{remark}
With a criterion for existence in hand, we proceed to state a criterion for uniqueness, justifying our focus on strict pseudoconvexity and strict pseudomonotonicity.
\begin{theorem}[Lemma 12.3, p. 516, \citep{hadjisavvas2006handbook}]
  \label{thm:uniqueness}
  Let $K\subset H$ be a non-empty subset of $H$. If $T$ is strictly pseudomonotone, then the problem $VI(T,K)$ has at most one solution.
\end{theorem}
