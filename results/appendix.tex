
%\section{Results on variational inequalities}
%\label{sec:appendix}
%The following section contains the necessary results on variational inequalities,

%We start by recalling the definition of pseudo-convexity
%The corresponding notion of convexity is
 %First, we need to introduce strict pseudomonotonicity.
%\begin{definition}
% The operator $T: K \to H$ is strictly pseudomonotone if for every pair $x\neq y$ we have
% \begin{align}
%   \ip{x-y}{Ty} \geq 0 \Rightarrow \ip{x-y}{Tx} > 0
% \end{align}
%\end{definition}
\begin{comment}
Proving strict pseudomonotonicity in itself can be hard, but thankfully the notions of strict pseudomonotonicity and strict pseudoconvexity are related.
\begin{theorem}[Theorem 12.13, p. 521 \citep{hadjisavvas2006handbook}]
  Let $\Omega \subset H$ be an open convex subset, and let $f:\Omega \to \R$ be Gateaux differentiable. Then $f$ is strictly pseudoconvex if and only if $\nabla f$ is strictly pseudomonotone.
\end{theorem}
\end{comment}

\begin{comment}

\citep[Theorem 3.6]{maugeri2009existence}, which we restate.
\begin{theorem}
  \label{thm:existence}
  Let $K\subset H$ be a closed convex set and $T:K\to H$ a pseudomonotone operator which is lower hemicontinuous along line segments, i.e. for all $x,y\in H$ the mapping $\xi \mapsto \ip{T\xi,}{x-y}$ is lower semicontinuous for $\xi \in \{\eta \in H : \eta = tx+(1-t)y, \quad t\in [0,1]\}$. Assume that there exists $u_0 \in K$ and $R> \norm{u_0}$ such that
  \begin{equation}
    \ip{Tv}{v-u_0} \geq 0, \forall v \in K \cap \{v \in H : \norm{v} = R \}
  \end{equation}
  then the variational inequality $VI(T,K)$ has a solution.
\end{theorem}
\end{comment}
%\begin{remark}
%  \label{rem:weak_compact}
%  Weak compactness of $K$ also ensures that $VI(T,K)$ has a solution \citep[Theorem 12.1, P. 510]{hadjisavvas2006handbook}, giving existence of solutions in the finite-dimensional case.
%\end{remark}
%With a criterion for existence in hand, we proceed to state a criterion for uniqueness, justifying our focus on strict pseudoconvexity and strict pseudomonotonicity.
%\begin{theorem}[Lemma 12.3, p. 516, \citep{hadjisavvas2006handbook}]
%  \label{thm:uniqueness}
%  Let $K\subset H$ be a non-empty subset of $H$. If $T$ is strictly pseudomonotone, then the problem $VI(T,K)$ has at most one solution.
%\end{theorem}
\section{Calculations}
\subsection{Existence}
\label{appendix:calculations}
We complete the omitted calculations from the main text. Initially,
\begin{equation}
  \label{eq:calc_app}
  \ip{-dU(\sigma_c,\sigma_p)}{(\sigma_c-1, \sigma_p-1)} = \ip{\frac{\sigma_c}{K_0 + K\phi} - 1}{\sigma_c
  - 1} + \ip{\frac{F_p\beta_p\sigma_p}{F_p + \ip{\sigma_p}{\sigma_c}}}{\sigma_c-1}
  - \ip{\frac{F_p^2\beta_p\sigma_c}{(F_p + \ip{\sigma_p}{\sigma_c})^2}}{\sigma_p - 1} + \ip{c \beta_p \sigma_p}{\sigma_p -1}
\end{equation}
Using that $\ip{1}{\sigma} = 1$, we can write out \Cref{eq:calc_app} and gather the positive and negative terms
\begin{equation}
  \begin{split}
  \ip{-dU(\sigma_c,\sigma_p)}{(\sigma_c-1, \sigma_p-1)} = \\
  \ip{\frac{\sigma_c}{K_0 + K\phi}}{\sigma_c} + \ip{c \beta_p \sigma_p}{\sigma_p} \\ + \ip{\frac{F_p\beta_p\sigma_p}{F_p + \ip{\sigma_p}{\sigma_c}}}{\sigma_c} - \ip{\frac{F_p^2\beta_p\sigma_c}{(F_p + \ip{\sigma_p}{\sigma_c})^2}}{\sigma_p} \\
  - \norm{c\beta_p \sigma_p}_1 - \norm{\frac{\sigma_c}{K_0 + K\phi}}_1 - \norm{\frac{F_p\beta_p\sigma_p}{F_p + \ip{\sigma_p}{\sigma_c}}}_1 \\
   - \norm{\frac{F_p^2\beta_p\sigma_c}{(F_p + \ip{\sigma_p}{\sigma_c})^2}}_1
\end{split}
\end{equation}
Remark that
\begin{equation}
  \ip{\frac{F_p\beta_p\sigma_p}{F_p + \ip{\sigma_p}{\sigma_c}}}{\sigma_c} \geq \ip{\frac{F_p^2\beta_p\sigma_c}{(F_p + \ip{\sigma_p}{\sigma_c})^2}}{\sigma_p}
\end{equation}
Defining $C_1 = \frac{1}{K_0 + K \ess \sup \phi}$ and $C_2 = c \ess \inf \beta_p$ we have:
\begin{equation}
  \begin{split}
  \ip{-dU(\sigma_c,\sigma_p)}{(\sigma_c-1, \sigma_p-1)} \geq C_1 \norm{\sigma_c}^2_2 + C_2 \norm{\sigma_p}_2^2 \\
  - \norm{c\beta_p \sigma_p}_1 - \norm{\frac{\sigma_c}{K_0 + K\phi}}_1 - \norm{\frac{F_p\beta_p\sigma_p}{F_p + \ip{\sigma_p}{\sigma_c}}}_1 \\
   - \norm{\frac{F_p^2\beta_p\sigma_c}{(F_p + \ip{\sigma_p}{\sigma_c})^2}}_1
  \end{split}
\end{equation}
Since $\norm{\sigma_i}_1=1, i \in \{c,p\}$, all terms involving $\norm{\cdot}_1$ are uniformly bounded, and can be gathered in a single uniformly bounded function $W$. Hence we end with:
\begin{equation}
    \ip{-dU(\sigma_c,\sigma_p)}{(\sigma_c-1, \sigma_p-1)} \geq C_1 \norm{\sigma_c}_2^2 + C_2 \norm{\sigma_p}_2^2 - W(\sigma_c,\sigma_p)
\end{equation}
%\subsection{Strict pseudomonotonicity}
