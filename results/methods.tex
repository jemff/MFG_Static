%\section{Methods}

\section{Population games: Mean-field games and monomorphism}


The essential assumption in population games is that we have  types of different agents playing a game against each other, where the agents of each type are internally indistinguishable. The payoff for a single individual depends on the behavior of the rest of the population, as well as the other populations and its own choice. The populations are assumed to be so large that the choice of a single individual does not change the mean.

Since all agents of each type are indistinguishable, when they all play the optimal strategy simultaneously, by interchangeability the Nash equilibrium must be that they follow the same strategy.

An alternative point of view for population games is assuming monomorphic populations from the get-go instead of as an emergent phenonenom \citep{vincent2005evolutionary}. There is a stark difference between the Nash equilibria of a game where the monomorphism is imposed and where it emerges, \citep{kvrivan2008ideal,collet2019algorithmic}. We quantify this explicitly:
\begin{proposition}
  \label{prop:doubleup}
  Consider a population game with a population of size $N$ with payoff $U$ given by a bilinear form $U$ specified by a matrix $A$ such that $U(\sigma_1,\sigma_2) = \ip{\sigma_1}{A \sigma_2}$. The expected payoff for a player with an inherently monomorphic population is twice that of a player in the mean-field game.
\end{proposition}
\begin{proof}
The payoff for a player with strategy $\sigma$ in a  monomorphic population of size $N$ is
\begin{align}
  U(\sigma,\sigma) = \ip{\sigma}{A N \sigma}
\end{align}
Conversely, if we assume that the relevant opposing strategy is the mean $\overbar{\sigma}$ of an a-priori polymorphic collection strategies the payoff is
\begin{align}
  U(\sigma, \overbar{\sigma}) = \ip{\sigma}{AN \overbar{\sigma}}
\end{align}
Since we assumed all agents are identical in the mean-field game, at the Nash equilibrium they must all follow the same strategy. Therefore, at the Nash equilibrium $\sigma = \overbar{\sigma}$, ie.
\begin{equation}
  \overbar{\sigma} = \argmax_{\sigma} U(\sigma,\overbar{\sigma})
\end{equation}
so we are interested in extrema where $\sigma=\overbar{\sigma}$.
Consider the Karush-Kuhn-Tucker (KKT) conditions at the Nash equilibrium for the monomorphic game
\begin{align}
  \begin{split}
    2AN\sigma + \mu - \lambda_1 \cdot 1 = 0 \\
    \ip{\sigma}{\mu} = 0 \\
    \mu \geq 0 \\
    \sigma \geq 0 \\
    \int_X \sigma - 1 = 0
  \end{split}
\end{align}
At every point with a non-zero concentration, the payoff for the monomorphic population is $\lambda_1$. We can likewise find the pointwise payoff for the mean-field game through the KKT-conditions:
\begin{equation}
  \begin{split}
    AN\sigma + \mu - \lambda_2 \cdot 1 = 0 \\
    \ip{\sigma}{\mu} = 0 \\
    \mu \geq 0 \\
    \sigma \geq 0 \\
    \int_X \sigma - 1 = 0
  \end{split}
\end{equation}
the payoff here is $\lambda_2 = 2\lambda_1$. In conclusion, the individual payoff for a player using the optimal strategy in a monomorphic population is twice that of an individual player in the mean-field game where the monomorphic behavior is an emergent property.
\end{proof}
As \Cref{prop:doubleup} shows, assuming a monomorphic population has a drastic effect on the expected payoff in a simple game, highlighting the necessity of considering monomorphism as an emergent property. Therefore, the approach advocated in \citep{vincent2005evolutionary} of optimizing the pr. capita growth directly in a population game overestimates the possible growth if there are density-dependent effects.


In \Cref{prop:doubleup} we also see the connection between the ideal free distribution and mean field games, where the classical ideal free distribution emerges as the mean-field equilibrium .

The proof of \Cref{prop:doubleup} suggests a connection between mean-field population games and games with monomorphic populations, allowing the study of mean-field games using the tools of monomorphic games.
\begin{definition}
  \label{def:correspondence}
  Consider a mean-field game with populations $N_1,\dots,N_M$ with individual payoffs specified by $U_i(\sigma_i, (\overbar{\sigma}_j)_{j=1}^M, (N_j)_{j=1}^M)$. Assume $\nabla_{\sigma_i}U_i\mid_{\sigma_i=\overbar{\sigma}_i}$ admits a primitive $U_{mon}$. We define the monomorphic equivalent game by the payoff functions $U^i_{mon}$.
\end{definition}
The game specified by the collection $U^i_{mon}$ is a monomorphic game, and in case the Nash equilibrium is unique, the Nash equilibria of the original game and the one specified by $U_i^{mon}$ are the same. Later on, by utilizing the tool-box of variational inequalities, we will give a sufficient criterion to check when the Nash equilibrium is unique, so \Cref{def:correspondence} may be used. The notion of a monomorphic equivalent game eases the analysis of mean-field games, and allows direct illustration of the difference between the emergent monomorphism in a mean-field game and imposed monomorphism.

As an example consider a habitat-choice game where a consumer in a total population of size $N$ with a growth rate given by a Type II functional response can either stay in a refuge or go into a foraging arena. In the foraging arena there is a population of predators. The predators also have a Type II functional response, but their satiation level depends on the mean time spent in the the arena by the consumers. If we use the instantaneous growth as a fitness proxy and let $\sigma$ modelling the time spent in the arena range from $[0,1]$, this situation is modeled by the payoff function:
\begin{align}
  U(\sigma,\overbar{\sigma}) = \frac{\sigma r}{\sigma c_1 + 1} - c_2\frac{\sigma}{1 + c_3 N \overbar{\sigma}}
  \label{eq:simple_mfg}
\end{align}
Using \Cref{def:correspondence}, we can see that \Cref{eq:simple_mfg} is equivalent to the monomorphic game
\begin{align}
  U^{mon}(\sigma,\sigma) = \frac{\sigma r}{\sigma c_1 + 1} - c_2\log(1+N c_3\sigma)
\end{align}
which contrasts the naive monomorphic game,
\begin{align}
  U(\sigma,\sigma) = \frac{\sigma r}{\sigma c_1 + 1} - c_2\frac{\sigma}{1 + N c_3\sigma}
\end{align}
As the growth of $c_2\frac{\sigma}{1 + c_3\sigma}$ is smaller than that of $c_2\log(1+c_3 \sigma)$, the maximum of $U(\sigma,\sigma)$ will be attained at a larger value of $\sigma$. This corresponds to an inherently monomorphic population being more risk-seeking, as individuals will sacrifice themselves for the greater good. This example, along with \Cref{prop:doubleup} lets us conjecture that a monomorphic population will generally be more risk-seeking.


\section{Nash equilibria and variational inequalities}
Calculating Nash equilibria is generally a hard problem. A fruitful approach to calculating Nash equilibria is via. the theory of complementarity problems and variational inequalities \citep{karamardian1969nonlinear,nabetani2011parametrized}. We unite the approach of variational inequalities and mean-field games, and by using relatively modern theory, we can characterize a rather general setup that guarantees uniqueness and existence of Nash equilibria for population games in both a finite and infinite dimensional setting.
\begin{definition}
  Define $H$ as the real Hilbert space $L^2(X)$, where $X$ is either a finite set of points or the union of the interval $[0,1]$ and a finite set of points. Define $H_+ \subset H$ as the a.e. positive functions in $H$.
\end{definition}
Finding Nash equilibria is one of the oldest applications of (linear) complementarity theory. We define a complementarity problem:
\begin{definition}
  Let $H$ be a real Hilbert space, and $K \subset H$ be a closed convex cone. Define $K^* = \{ x \in H : \ip{x}{y} = 0, \quad \forall y\in K\} $. Assume $T:K \to H$. The complementarity problem $CP(T,K)$ is the problem of finding an element $x^*$ such that
  \begin{equation}
    \begin{split}
    \ip{x^*}{Tx^*} = 0 \\
    Tx \in K^*, \quad  x\in K
  \end{split}
  \end{equation}
\end{definition}
However, showing existence and uniqueness for complementarity problems is generally hard, which is where their relationship to variational inequalities comes in handy.
\begin{definition}
  Let $H$ be a real Hilbert space and $K\subset H$ be a non-empty subset of $H$. Let $T: K \to H$. The variational inequality $VI(T,K)$ is the following system
  \begin{equation}
    x \in K, \ip{y-x}{Tx} \geq 0, \quad \forall y \in K
  \end{equation}
  and a solution is an $x$ satisfying the inequality.
\end{definition}
If $K$ is a convex cone, the problems $VI(T,K)$ and $CP(T,K)$ have the same solutions, which can be seen by setting $y=2x$ and $y=0$ respectively.

To motivate the approach based on complementarity and variational inequalities, assume that we have a game with $M$ players and payoff functions $U_i$ and strategies $\sigma_i$. At the Nash equilibrium, each payoff function $U_i$ must satisfy the Karush-Kuhn-Tucker (KKT) conditions with respect to $\sigma_i$.
\begin{align}
  \nabla_{\sigma_i}U_i((\sigma_j)_{j=1}^N) + \mu_i - \lambda \cdot 1 = 0
  \ip{\sigma_i}{\mu_i} = 0
  \mu_i \geq 0
  \sigma_i \geq 0
  \int_X \sigma_i - 1 = 0
\end{align}
The Nash equilibrium of the game specified by the family $(U_i)$ corresponds to finding a system $\sigma_i^*$ satisfying the KKT conditions simultaneously for every pair $U_i$, \citep{deimling2010nonlinear}. Thus the total condition for a Nash equilibrium is:
\begin{equation}
  \begin{split}
  \begin{pmatrix}
    \nabla_{\sigma_1} U_1 \\
    \vdots \\
    \nabla_{\sigma_N} U_N
\end{pmatrix} + \begin{pmatrix}
    \mu_1 \\
    \vdots \\
    \mu_N
\end{pmatrix} + \begin{pmatrix}
    \lambda_1 \cdot 1_k \\
    \vdots \\
    \lambda_N \cdot 1_k
\end{pmatrix} = 0 \\
\ip{
\begin{pmatrix}
  \mu_1 \\
  \vdots \\
  \mu_N
\end{pmatrix}}{ \begin{pmatrix}
  \sigma_1 \\
  \vdots \\
  \sigma_N
\end{pmatrix}} = 0 \\
\mu_i \geq 0 \\
  \sigma_i \geq 0 \\
    \int \sigma_i dx = 1
    \label{eq:KKT_total}
  \end{split}
\end{equation}
To simplify \Cref{eq:KKT_total}, we define $dU = \begin{pmatrix} \nabla_{\sigma_1} U_1 \\
    \vdots \\
    \nabla_{\sigma_N} U_N \end{pmatrix}$, $S=(\sigma_i)_{i=1}^M$, and $\lambda = (\lambda_i)_{i=1}^M$. Then the problem can be restated as
\begin{equation}
  \label{eq:complementarity_form}
  \ip{dU(S)+\lambda}{S} = 0 \\
  \ip{S_i-1}{1}\lambda = 0
  -dU(S)-\lambda \geq 0 \\
  \ip{S_i-1}{1} \in \{0 \}
  S\geq 0
  \lambda \in \R
\end{equation}
which is a complementarity problem with $K = (H_+ \osum \R)^n$, $K^* = (H_+ \osum \{0\})^n$ and
\begin{equation}
  T(S,\lambda)=\begin{pmatrix} dU(S) + \lambda \\ (\ip{\sigma_i}{1} - 1)_{i=1}^M \end{pmatrix}
\end{equation}
If we do not include the criterion $\int \sigma_i dx = 1$ explicitly, but instead consider the convex set $K=\{f \in H: \int f dx = 1\}$. Then we are looking for a vector $S = (\sigma_i)_{i=1}^M$ that minimizes
\begin{equation}
  \ip{dU(S)}{S-T} \geq 0 \forall T\in K^n
\end{equation}
which is more tractable for theoretical considerations, while the complementarity formulation is preferable for solving the problem numerically.

In case the payoff-functions $\sigma_i$ are sufficiently nice, the machinery of variational inequalities can be applied to show existence and uniqueness of the Nash equilibrium. Asking for strict convexity is too strong, but we can relax the requirement to the related weaker notion of strict pseudoconvexity. See \Cref{sec:appendix} for further details.
We can gather the results on pseudoconvex functions and their application to game theory in the following theorem:
\begin{theorem}
  Cnosider a game with $N$ players with strictly pseudoconvex payoff functions $U_i$ and strategies $\sigma_i$ in $H$. The game has a unique Nash equilibrium if
  \begin{equation}
    dU=
    \begin{pmatrix}
      \nabla_{\sigma_1} U_1\\
      \vdots \\
      \nabla_{\sigma_N} U_N
    \end{pmatrix}
  \end{equation}
  satisfies the criterion of \Cref{thm:existence} or $H$ is finite dimensional.
\end{theorem}
\begin{proof}
   If $\nabla_{\sigma_i} U_i$ is strictly pseudomonotone, then so is their direct sum $dU$ by linearity of the inner product. By \Cref{thm:uniqueness} this solution is unique since $dU$ is strictly pseudomonotone. So if the solution exists, it is unique. By assumption \Cref{thm:existence} gives existence of a solution of $VI(dU,K)$ in case $H$ is infinite dimensional. If $H$ is finite-dimensional then $K$ is compact and there exists a solution \Cref{rem:weak_compact}.
\end{proof}
With this result, we can show that there exist unique fixed points of population games with strictly pseudoconvex payoff functions and bounded dynamics.
\begin{theorem}
  \label{thm:pop_game_exists_unique}
  A population game with $N$ populations and dynamics specified by $f_i$, with strictly pseudoconvex functions payoff $U_i$ and strategies $\sigma_i$ has a unique fixed point with a unique Nash equilibrium if $f_i/x_i$ are strictly pseudomonotone and satisfy \Cref{thm:existence}, and the set of fixed points of $f_i$ is uniformly bounded in $\sigma$.
\end{theorem}
\begin{proof}
  The game specified by $\sigma_i$ has a unique Nash equilibrium for each $x\in \R^N$, given by a complementarity problem over $K$ with operator specified by $dU = (\nabla_{\sigma_i} U_i)_{i=1}^N$. The solution specifies a continuous mapping $G: K^N \to \R^N$, \cite[Theorem 4.2]{barbagallo2009continuity}. Likewise, due to the assumptions on $f_i$ the complementarity problem $\ip{(f_i)_{i=1}^N(x)}{x}$ has a unique solution for every $\sigma$. Denote this solution function by $F$.

  Finding a fixed point of the dynamical system along with a Nash equilibrium then corresponds to finding a fixed point of $F\circ H: \R_+^N \to \R_+^N$. Since $F$ is assumed to be uniformly bounded, hence has compact range, $F\circ H$ has compact range. Therefore $F\circ H$ has a fixed point by Schauders fixed point theorem.

  For uniqueness, note that for each vector $\sigma$, the variational inequality
  \begin{equation}
    x \in \R_+^n, \ip{(f_i/x_i)_{i=1}^N(x)}{y-x}\geq 0 \quad y\in \R_+^n
  \end{equation}
  has a unique solution, by strict pseudomonotonicity of $f_i/x_i$. If we append the two operators $(f_i)_{i=1}^N$ and $\nabla_{\sigma_i} U_i$, we get a single strictly pseudomonotone variational inequality, where there exists a unique solution, since all terms are strictly pseudomonotone.
 Therefore the simultaneous problem
\begin{equation}
  \begin{split}
  S\in K^n, \ip{dU(S))}{S'-S} \geq 0, \quad S' \in K^n \\
  x \in \R_+^n, \ip{(f_i/x_i)_{i=1}^N(x)}{y-x}\geq 0 \quad y\in \R_+^n
\end{split}
\end{equation}
  has a unique solution, showing the existence and uniqueness of a fixed point for a population game.
\end{proof}
Remark that the assumption of uniformly bounded fixed-point space as a function of the strategies is biologically reasonable, since e.g. Type II growth functions are uniformly bounded and a metabolic loss is usually constant.

Having established the general theory for Nash equilibria, and demonstrated how to handle mean-field games as monomorphic games, we are ready to apply the theory to a Rosenzweig-MacArthur system with fast adaptive behavior.
\section{Revisiting the Rosenzweig-MacArthur model}
\label{sec:model_rm}
%Having established the general framework, we can proceed to defining the concrete model of interest.
We consider a predator-prey system modeled as a Rosenzweig-MacArthur system in a spatially heterogeneous habitat, where the predators are specialized so that their hunting success is maximal in the most productive zone of the habitat.
We assume that predators $(P)$ and consumers $(C)$ share a heterogeneous habitat, modeled as the interval $[0,1]$. The mean strategy of the consumer population is $\overbar{\sigma}_c$ and the mean strategy of the predator population is $\overbar{\sigma}_p$.
The population dynamics are given by a behaviorally modified Rosenzweig-MacArthur system. The predator clearance rate is $\beta_p = \beta_{l} + \beta_0$ where $\beta_l$ varies locally and $\beta_0$ is the minimal clearance rate. The function $\beta_p$ is normalized to a constant value of 5.
Consumption events are assumed local, so the expected encounter rate between predators and prey is $\ip{\beta_p \overbar{\sigma}_p}{\overbar{\sigma}_c}$. To model intraspecific predator competition, we have added a term $c\ip{\beta \overbar{\sigma}_p}{\overbar{\sigma}_p}$ where $c$ describes the level of competition. The carrying capacity is given by $K\phi + K_0$,
 where $K$ varies, $\phi$ is a distribution, and $K_0$ is the minimal carrying capacity. Denoting the maximal predator growth rate by $F_p$, the consumer clearance rate by $\beta_c$, the dynamics are given by:
%Inserting \Cref{eq:r_ss} in \Cref{eq:bas_dyn}, we arrive at the population dynamics:
\begin{align}
  \dot{C} &= f_c = C \pa{\ip{\beta_c \overbar{\sigma}_c}{1-\frac{\beta_c}{K\phi + K_0}\overbar{\sigma}_c C} - \frac{F_p \ip{\beta_p \overbar{\sigma}_c}{\overbar{\sigma}_p} P}{F_p + \ip{\beta_p \overbar{\sigma}_c}{\overbar{\sigma}_p} C} } \\
  \dot{P} &= f_p = P \pa{\epsilon \frac{F_p \ip{\beta_p \overbar{\sigma}_c}{\overbar{\sigma}_p} C}{F_p + \ip{\beta_p \overbar{\sigma}_c}{\overbar{\sigma}_p} C} - c \ip{\overbar{\sigma}_p}{\beta_p \overbar{\sigma}_p}  - \mu_p}
  \label{eq:dynamics}
\end{align}
Introducing the quality ($q$) of the habitat as a parameter, the functions $\beta_p$ and $\phi$ are given by
\begin{align}
  \beta_p &= 5 \frac{\exp(-(q x)^2) + \beta_0}{\int_0^1 \exp(-(q x)^2) + \beta_0 dx} \\
  \phi &= \frac{\exp(-q x) + K_0}{\int_0^1 \exp(-q x) + K_0 dx}
\end{align}
where the normalization is to keep a constant total encounter rate and carrying capacity of $K$ when varying $q$. The choice of functions reflect a predator specialized in hunting the most productive zones, and a heterogeneous productivity.

\subsection{Existence and uniqueness of the equilibrium}
We model predator and consumer movement as instantaneous, with each predator and consumer seeking to maximize its fitness at each instant. The fitness of an individual depends on the mean strategy of its own type and that of either the predators, or consumers, respectively. Denoting the individual consumer and predator strategies by $\sigma_c$ and $\sigma_p$ respectively, the instantaneous fitness $U_c$ of a consumer and a predator $U_p$ are:
\begin{align}
  U_c(\sigma_c, \overbar{\sigma}_c, \overbar{\sigma}_p) &= \ip{\beta_c \sigma_c}{1-\frac{\beta_c}{K\phi + K_0}\overbar{\sigma}_c C} - \frac{F_p \ip{\beta_p \sigma_c}{\overbar{\sigma}_p} P}{F_p + \ip{\beta_p \overbar{\sigma}_c}{\overbar{\sigma}_p} C} \\
  U_p(\sigma_p, \overbar{\sigma}_c, \overbar{\sigma}_p) &= \epsilon \frac{F_p \ip{\beta_p \overbar{\sigma}_c}{\sigma_p} C}{F_p + \ip{\beta_p \overbar{\sigma}_c}{\sigma_p} C} - c \ip{\sigma_p}{\beta_p \overbar{\sigma}_p}  - \mu_p
\end{align}
Denoting the optimum by $^*$, at the Nash equilibrium $\sigma_c^*=\overbar{\sigma}_c$ and $\sigma_p^* = \overbar{\sigma}_p$. We use the method of \Cref{def:correspondence} to convert both mean-field payoffs into a monomorphic equivalent form \Cref{def:correspondence}
\begin{align}
  \label{eq:mon_eq_forms}
  U_c^{mon}(\sigma_c, \sigma_p) &= \ip{\beta_c \sigma_c}{1-\frac{1}{2}\frac{\beta_c}{K\phi + K_0 2}\sigma_c C} - P\log\pa{F_p + \ip{\beta_p \sigma_c}{\sigma_p} C} \\
  U^{mon}_p(\sigma_p, \sigma_c) &= \epsilon \frac{F_p \ip{\beta_p \overbar{\sigma}_c}{\sigma_p} C}{F_p + \ip{\beta_p \sigma_c}{\sigma_p} C} - \frac{1}{2}c \ip{\sigma_p}{\beta_p \sigma_p}  - \mu_p
\end{align}
We need to verify that $-U_c^{mon}$ and $-U_p^{mon}$ are strictly pseudoconvex, as well as show existence. To this end, we state a charactirization of strict pseudomonotonicity for differentiable functions.
\begin{lemma}
  A function $f: K \subset H \to \R$ is strictly pseudoconvex if
  \begin{equation}
    \ip{f(x)}{u} = 0 \Rightarrow \ip{(\nabla_x f(x))h}{h} > 0
  \end{equation}
\end{lemma}
A proof can be found in \citep[Proposition 2.8, p.96]{hadjisavvas2006handbook}
\begin{proposition}
  The functions $-U_c^{mon}$ and $-U_p^{mon}$ are strictly pseudconvex, and their derivatives $\begin{pmatrix}-\nabla_{\sigma_c} U_c^{mon} \\ -\nabla_{\sigma_p} U_p^{mon}\end{pmatrix}$ satisfy the criteria of \Cref{thm:existence}.
\end{proposition}
\begin{proof}
  We start by showing the strict pseudoconvexity. The function $-U_p^{mon}$ is strictly convex, as it has strictly positive second derivative, so the derivative with respect to $\sigma_p$ is strictly monotone, therefore strictly pseudomonotone. Note that on $K$ a function $g$ is strictly pseudomonotone if and only if $g+\lambda$ is strictly pseudomonotone for any $g\in \R$, since $\ip{g(x)+\lambda}{x-y}=\ip{g(x)}{x-y}+\lambda \int x d\mu + \lambda \int y d\mu=\ip{g(x)}{x-y}$. If we consider $f(x)=(-\nabla_{\sigma_c} U_c^{mon}(x))-1$ and assume $\ip{f(x)}{u} = 0$, then
  \begin{equation}
    \ip{x}{u}\frac{C \beta_c }{K\phi + K_0} + \frac{\ip{\sigma_p}{h}}{F_p + \ip{\sigma_p}{x}} = 0
  \end{equation}
  Since $K$ consists of positive functions, $\ip{x}{u} = 0$ and $\ip{\sigma_p}{h} = 0$. If we consider
  \begin{equation}
    H(x,u)=\ip{(\nabla_x f)(x)u}{u} = \ip{u}{u}\frac{ C \beta_c }{K\phi + K_0} - \frac{\ip{\sigma_p}{u}\ip{\sigma_p}{h}}{F_p + \ip{\sigma_p}{x}}
  \end{equation}
  then $H(x,u)>0$ since $\ip{\sigma_p}{u}=0$, so $S$ is strictly pseudomonotone.

  To show that there exists a solution, start by noting that for all $h\in H^2$, the mapping $S \mapsto \ip{dU(S)}{w}$ is continuous in $S$, hence lower semi-continuous, hence also lower-semicontinuous on line segments.
  For the technical criterion of \Cref{thm:existence} pick $u_0 = [1,1]$ and consider $R=1+\epsilon$, $\epsilon>0$. Then we wish to show that there exists $\epsilon$ such that
  \begin{equation}
    \begin{split}
    \ip{-dU(\sigma_c,\sigma_p)}{(1-\sigma_c, 1-\sigma_p)} \geq 0 \\
    \ip{dU(\sigma_c,\sigma_p)}{(\sigma_c-1, \sigma_p-1)} \geq 0
  \end{split}
  \end{equation}
  for $\norm{(\sigma_c,\sigma_p)} = 1+\epsilon$.
  It is sufficient to show that there exists $\epsilon$ such that each term in the inner product is positive. We start with the consumer term, where we utilize that $\ip{\sigma_c}{1}=1$
  \begin{equation}
    \begin{split}
    1-\ip{\sigma_c}{\sigma_c-1}\frac{ C \beta_c }{K\phi + K_0} - \frac{\ip{\sigma_p}{\sigma_c-1}}{F_p + \ip{\sigma_p}{\sigma_c}} \\
     =1-(1-\norm{\sigma_c}^2)\frac{ C \beta_c }{K\phi + K_0}+(1-\ip{\sigma_p}{\sigma_c})\frac{1}{F_p + \ip{\sigma_p}{\sigma_c}}
  \end{split}
  \end{equation}
  The terms $1-\norm{\sigma_c}^2$ and $1-\ip{\sigma_p}{\sigma_c}$ are uniformly bounded above by $\epsilon_1^2+2\epsilon_1$. If we define $\xi = \frac{C \beta_c }{K\phi + K_0}$ and $\eta = \frac{1}{F_p + \ip{\sigma_p}{\sigma_c}}$, $\epsilon_1$ can be determined by solving the inequality
  \begin{equation}
    (\epsilon_1^2+2\epsilon_1)(\xi+\eta) \leq 1
  \end{equation}
  Proceeding in the same fashion with the second term, we arrive at a pair of constants $\epsilon_1,\epsilon_2$. We can then pick the minimum of these two, showing the desired.
\end{proof}
Since we are also interested in the fixed-points of the population dynamics \Cref{eq:dynamics}, we are also interested in showing that the fixed-point of the dynamics exists and is unique. Here we can directly use the theory we have developed.  The functions $f_1/C$ and $f_2/P$ are strictly monotone, and the set of fixed-points is uniformly bounded due to the metabolic terms and logistic terms, respectively. Therefore, by \Cref{thm:pop_game_exists_unique} there is a unique equilibrium for the population game.


%By rephrasing the population into a varitaional inequality, we unlock the full arsenal of tools available for the solution of variational inequalities.

%We formulate the optimization problem as a feasibility problem via. the optimization framework casadi \citep{}. To solve the problems the solver IPOPT \citep{} is invoked, using the Haswell linear algebra routines \citep{}. This approach to solving variational inequalities is state-of-the-art, and the solver was both robust and fast.
%Keywords: Casadi, IPOPT, state-of-the-art, simple to expand and scalable.

%\begin{enumerate}
%  \item
%    Introduce the theory of Variational inequalities, simple game
%  \item
%    Introduce mean field game, modify predator-prey to the mean field case
%  \item
%    Introduce continuous setting
%\end{enumerate}
