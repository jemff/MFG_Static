\section{Discussion and conclusion}
We study study population games, based on optimizing individual growth, modifying the definition of \citep{vincent2005evolutionary}. This is done through the introduction of mean-field games, which we show generalizes the ideal free distribution \citep{fretwell1969territorial}. We demonstrate that assuming a monomorphic population is not a viable alternative to the mean-field approach. We do this by showing that the pr. capita payoff a monomorphic population is twice what could be expected from playing the field  \citep{parker1978searching}. Hence, though a population of animals may all be indistinguishable, and appear to follow the same behavioral strategy, it is important to consider how this monomorphism emerges.


We establish existence and uniqueness of Nash equilibria for a large class of games using variational inequalities. In particular, we are able to handle non-convex payoff functions and continuous strategy spaces. Having determined existence and uniqueness of Nash equilibrium for the instantaneous game, we showed the existence and uniqueness of fixed-points for suitably nice population games. This provides a simple criterion for population games, extending theorems based on specific models \citep{cressman2010ideal,sandholm2010population}. It would be nice to prove that the population dynamics with optimal behavior are stable using general tools. A possible way to prove stability generally could be drawing on the theory of dynamical variational inequalities \citep{adly2018variational,brogliato2020dynamical, tang2020differential} or studying dynamical systems associated to bi-level variational inequalities \citep{anh2021dynamical}. Both of these fields are quickly developing, and a host of general stability, uniqueness and existence results have been published in recent years. Hence it seems plausible that in the near future results from e.g. differential variational inequalities will allow a general proof of the stability of systems with fast optimal behavior. This will provide a proof of the general observation that optimal behavior acts to stabilize systems \citep{valdovinos2010consequences}.



 %Our focus has been on population games, but the results on existence and uniqueness of Nash equilibria in mean-field games are generally applicable to games in biology.

We demonstrate the utility of our results by applying them to study a Rosenzweig-MacArthur system with fast optimal behavior. We easily establish existence and uniqueness of Nash equilibria, both for only consumers or predators and when both have optimal behavior. The method of proof is computational and combinatorial in nature, and hence can almost certainly be extended to larger more complex ecosystems where the Nash equilibrium appears unique but has not been shown to be unique \cite{pinti2019trophic}. This shows that our general results open up the study of much more general population games from a rigorous viewpoint than has otherwise been the case, \citep{cressman2010ideal, kvrivan2013behavioral, kvrivan2009evolutionary, broom2013game}.

Having shown that the Nash equilibrium exists and is unique, we show another of our main results, namely that the there is a unique stable population equilibrium. This is a major advance, drawing on both our results on population games and a topological argument. This adds to the small catalogue of population games where the population dynamics are provably stable with a unique fixed point \citep{kvrivan2009evolutionary}. The result also highlights the importance of considering the interplay between the underlying game and the dynamical system, since we utilize the strict dominance of the payoff functions over the vector-field specifying the sign of the population dynamics.

 %WRITE MORE HERE!!!! THIS IS TOPIC IN RAPID DEVELOPMENT, MANY ARTICLES RECENT YEARS!! DVI ETC STARTED WITH PANG STEWART, MUCH HAS BEEN DONE ALSO NE


After showing existence and uniqueness, we analyzed the modified Rosenzweig-MacArthur game numerically by discretizing space. Adding optimal individual behavior appears to eliminate the paradox of enrichment \citep{rosenzweig1971paradox}, which is a common consequence of optimal behavior in ecosystem models \citep{abrams2010implications, valdovinos2010consequences}. However, we were unable to show why this happened. In the sensitivity analysis we saw that the the intraspecific predator competition did not noticeably affect the predator population levels, while elevating the consumer population levels, which was surprising \citep{abrams2010implications}. The increase in carrying capacity increased both predator and prey levels, as is usually the case in models with optimal behavior \citep{valdovinos2010consequences}. The sensitivity analysis also showed the emergence of an interesting pattern of consumer predator co-existence, with an ideal-free distribution emerging in the areas without any predators.



%Skriv noget om sensitvities analysen?
We have not touched on the topic of differential games, where the payoff function for an individual depend on e.g. the entire life history. We expect that the use of variational inequalities and complementarity problems can be extended to differential games. For instance by using Pontryagins maximum principle and approaching the game as a differential variational inequalities \citep{pang2008differential}. Alternatively, the Nash equilibrium can be found explicitly at every time-step in for the Hamilton-Jacobi-Bellman equation when doing backwards time-stepping.



%Overall our most important finding is that playing the field can be incorporated in complex population games with instantaneous behavior, and that the resulting Nash equilibrium exists and is unique across a wide range of circumstances.
  %We show the framework of mean-field games is applicable in ecology through multi-species population games with optimal individual behavior. We establish theoretical results for existence and uniqueness of Nash equilibria with continuous strategy spaces.

%The payoff for an individual in a large population of indistinguishable individuals is shown much higher when all individuals are a-priori assumed to follow the same strategy.



%We show that a behavorially modified Rosenzweig-MacArthur model has a unique equilibrium by drawing on the general theory we developed. This opens up for sensitivity analysis of the Rosenzweig-MacArthur with respect to the carrying capacity, and the intraspecific predator competition. This analysis is facilitated by a numerical approach based on complementarity. Posing the problem as a complementarity problem changes an optimization problem into a feasibility problem, which is handily solved by IPOPT using the HSL subroutines. The runtimes were impressive, even with a discretization with several hundred grid points.





%KKT can be reformulated as VI, developed by stampacchia mangasarian goeleven et al
%VI solvable numerically, NE existence gives that minimum exists and is 0
%Wide class of VI unique solution, applicable?
%Write up equations for two players, remark that the same for N

%\begin{enumerate}
%  \item What did we find?
%    Highlighted difference between individual and group-level Optimization
%    Concrete calculations showing that individual optimization is less efficient than group level
%    Showed existence and uniqueness of Nash equilibrium
%    Showed existence and uniqueness of population game equilibrium
%    Demonstrted on behavorially modified Rosenzweig-MacArthur model
%    Fast and efficient numerical approach
%  \item What do our results mean?
%    we now have good criteria for uniqueness and existence, less ad-hoc.%
    %Behavior can become the norm.
    %Models need to look at the individual perspective rather than population level.
    %Illustrates the difference between emergent monomorphism and inherent monomorphism
    %Selfish behavior is much less advantageous for the collective

%  \item Perspectives in VI for NE in ecology  Compare to literature also the stuff by Patrizio. remark well-known that worse for species than population-thingy
%    Fast numerics can be implemented by anynone, for a fairly general set of games not just related to populations
%    Feasible to implemement population games generally, behavior can be as much of a choice as seasons
%    Realistic models of ecosystems
%    The work is hinted at in the work of Sandholm, but not developed further.
%    We focus on continuous, but much stronger results actually hold in finite habitat choice setting.
%    It has become possible to include bona-fide continuous habitats.
%  \item
%    Games with long-term optimization can be implemented as differential variational inequalities
%    Multiple species in actual Applications
%    Check the Scalability
%    Try wtih dedicated software, see how fast it can get.
%
%\end{enumerate}
