\section{Discussion and conclusion}
We

%KKT can be reformulated as VI, developed by stampacchia mangasarian goeleven et al
%VI solvable numerically, NE existence gives that minimum exists and is 0
%Wide class of VI unique solution, applicable?
%Write up equations for two players, remark that the same for N

\begin{enumerate}
  \item What did we find?
    Highlighted difference between individual and group-level Optimization
    Concrete calculations showing that individual optimization is less efficient than group level
    Showed existence and uniqueness of Nash equilibrium
    Showed existence and uniqueness of population game equilibrium
    Demonstrted on behavorially modified Rosenzweig-MacArthur model
    Fast and efficient numerical approach
  \item What do our results mean?
    we now have good criteria for uniqueness and existence, less ad-hoc.
    Behavior can become the norm.
    Models need to look at the individual perspective rather than population level.
    Illustrates the difference between emergent monomorphism and inherent monomorphism
    Selfish behavior is much less advantageous for the collective

  \item Perspectives in VI for NE in ecology  Compare to literature also the stuff by Patrizio. remark well-known that worse for species than population-thingy
    Fast numerics can be implemented by anynone, for a fairly general set of games not just related to populations
    Feasible to implemement population games generally, behavior can be as much of a choice as seasons
    Realistic models of ecosystems
    The work is hinted at in the work of Sandholm, but not developed further.
    We focus on continuous, but much stronger results actually hold in finite habitat choice setting.
    It has become possible to include bona-fide continuous habitats. 
  \item
    Games with long-term optimization can be implemented as differential variational inequalities
    Multiple species in actual Applications
    Check the Scalability
    Try wtih dedicated software, see how fast it can get.

\end{enumerate}
\subsection*{Habitat selection: Continuous and discrete}
There are fundamentally two different types of habitat selection that an animal can undertake. One is moving from one habitat to another, totally disconnected habitat, eg. by migration \citep{}, and another is choosing exactly the placement in a continuous setting such as a water column or landscape \citep{}. The two are often conflated, \citep{patriziopaper, jerome}, but they are fundamentally different, and care must be taken to distinguish between the situations. Picking a discrete probability distribution over a finite set and a continuous distribution are radically different and can lead to different outcomes, since neighbors are modelled as being infinitely far away in the discrete setting. The toolbox of Nash equilibria and mean-field games we have introduced works equally well over finite sets and continuous spaces, allowing a direct comparison \citep{}.
\todo[inline]{Maybe move this to the discussion section...}
\subsection{Variational inequalties}
Optimal control
<
