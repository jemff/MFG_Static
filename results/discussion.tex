\section*{Discussion}
%KKT can be reformulated as VI, developed by stampacchia mangasarian goeleven et al
%VI solvable numerically, NE existence gives that minimum exists and is 0
%Wide class of VI unique solution, applicable?
%Write up equations for two players, remark that the same for N

\begin{enumerate}
  \item What did we find?
  \item What do our results mean?
  \item Perspectives in VI for NE in ecology  Compare to literature also the stuff by Patrizio. remark well-known that worse for species than population-thingy
  \item MFG as approach for games, refer to papers where the distinction is not clear, note that Krivan and Cressman write it is the correct way
  \item Remark taht Sandholm is reinvinting field of VI - badly -
  \item Scalability?
  \item Generalize to "True" continuous time MFG
\end{enumerate}
\subsection*{Habitat selection: Continuous and discrete}
There are fundamentally two different types of habitat selection that an animal can undertake. One is moving from one habitat to another, totally disconnected habitat, eg. by migration \citep{}, and another is choosing exactly the placement in a continuous setting such as a water column or landscape \citep{}. The two are often conflated, \citep{patriziopaper, jerome}, but they are fundamentally different, and care must be taken to distinguish between the situations. Picking a discrete probability distribution over a finite set and a continuous distribution are radically different and can lead to different outcomes, since neighbors are modelled as being infinitely far away in the discrete setting. The toolbox of Nash equilibria and mean-field games we have introduced works equally well over finite sets and continuous spaces, allowing a direct comparison \citep{}.
\todo[inline]{Maybe move this to the discussion section...}
\subsection{Variational inequalties}
Optimal control
