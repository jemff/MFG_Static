\section*{Introduction}

%Mention a few classic papers, mention the problem.
Game theory is the natural tool to model the behavior of animals, who must respond to the behavior of other animals as well as complex and rapidly shifting environments. The application of game theory game theory, has led to theoretical strides in understanding the complex interplay between behavior and population dynamics \cite{cressman2010}.  Applying game-theory to patch-choice models gives an understanding of why animals are not all clustered at the best spot, \citep{cressman2004ideal, abrams2007role}. Even a simple Lotka-Volterra predator-prey system with optimal behavior changes radically, becoming asymptotically stable instead of oscillating, \citep{kvrivan2007lotka}. A game theoretical approach has been fruitful in studying simple ecosystems under the assumption of a few monomorphic populations or limited choices, but real-life behavorial choice involves choosing between a wide array of different strategies, in the context of intra-specific competition and outside risk. Constructing robust models that can handle such systems would allow for an integration of ecosystem function and behavior, a long-standing ambition \citep{schmitz2008individuals}.

We believe the way optimal behavior is included in studies of ecosystem function is usually oversimplified, due to computational and modelling limitations. A common simplification is to assume that at least one payoff is linear in the choice of strategy, \citep{genkai2007macrophyte}. There are computational and conceptutal advantages to studying linear models, and a lot of insight can be gleaned from them, \citep{cressman2010}, but the non-linear effects in natural systems are often signifcant. Models framed in terms of individual choice involving a non-linear payoff, usually work with implicit assumption that the optimization happens with the prior knowledge that all other members will follow the same strategy, by optimizing the raw pr. capita growth, \citep{malone2020ecology,stump2017optimally}. We believe that the correct approach is to assume that each animal optimizes individually, only taking into account the mean behavior of all other individuals, as in the ideal free distribution. Assuming very large populations, this situation can be modeled as a static mean-field game, the generalization of the ideal free distribution, where all animals of a single type follow the mean-field strategy at the Nash equilibrium.

We demonstrate an efficient numerical approach to finding Nash equilibria in games with both linear and non-linear payoffs, handling the mean-field case as a natural side-effect. The example we work from is a classic predator-prey game. We illustrate each aspect of our approach by gradually increasing the complexity of the model, from a simple Lotka-Volterra model with a refuge to a predator-prey system with Type II functional responses in a continuous habitat. The theoretical underpinnings of our method is the well-developed theory of finding Nash equilibria by recasting the problem as a variational inequality, which has had limited applications in biology. Having developed the general method, we use it to examine the population dynamics and the patch distribution at equilibrium as a function of changing refuge quality, intraspecific predator competition and

Add references to krivan and cressmans other articles, including the empirical one by krivan.
