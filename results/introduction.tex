\section{Introduction}

%Mention a few classic papers, mention the problem.
Game theory is the natural tool to model the behavior of animals, who must respond to the behavior of other animals as well as complex and rapidly shifting environments. A classical application of game-theory is patch-choice models, where the ideal free distribution emerges to explain spatial distributions  \citep{cressman2004ideal, fretwell1969territorial}. A game theoretical approach has been fruitful in studying simple ecosystems under the assumption of a few monomorphic populations or limited choices, but real-life behavorial choice involves choosing between a wide array of different habitats, in the context of intra-specific competition and outside risk. Models that can handle such systems allows for better models of habitat distribution, and represent a significant step forward in understanding natural systems \citep{morris2003shadows}.

A common simplification when including behavior in population models is to assume that at least one payoff is linear in the choice of strategy, \citep{krivan1997dynamic}. Linear models are suffficient to explain simple predator-prey dynamics with optimal behavior, \citep{kvrivan2007lotka}, but non-linear effects in natural systems are significant \citep{gross2009generalized}. The population game approach advanced by \citep{vincent2005evolutionary} for merging optimal behavior with population dynamics through maximizing the pr. capita growth implicitly assumes monomorphic populations, where all individuals intrinsically act as one \citep{malone2020ecology,stump2017optimally}. We propose a modification of the approach from \citep{vincent2005evolutionary}, based on individual optimization. We assume each invidiual maximies its own instantaneous growth rate, with density dependent behavior depending on the mean population behavior. This marks a return to the ideas of playing the field, \citep{smith1982evolution, cressman2010ideal}. If we assume that the animals of each type are indistinguishable and large populations, the game can be modeled as a static mean field game with multiple types.


The evolution of mean-field games has followed two parallel tracks, one in mathematical biology through the ideal free distribution \citep{fretwell1969territorial, cressman2004ideal, kvrivan2008ideal, cressman2010ideal}, and the other in mathematical optimization based directly on anonymous actors \citep{lasry2007mean, aumann1964markets, blanchet2016optimal}. The main focus in ecology has been on and specific families of games in depth \cite{broom2013game}, while the focus in mathematical optimization has been in establishing uniqueness and existence of Nash equilibria through the toolbox of variational inequalities \citep{karamardian1969nonlinear, gabay1980uniqueness, nabetani2011parametrized}. We demonstrate that in certain cases, a mean-field game can be rephrased equivalently as a normal-form game. This allows us to bring the entire toolbox of variational inequalities to bear on population games.


Using the theory of variational inequalities, we show that population games based on individual optimization have a unique equilibrium under very general assumptions. Our approach allows us to handle both continuous and discrete strategy spaces, but more technical assumptions are required for existence in the continuous setting. The classical ideal free distribution emerges as a special case of our approach, providing a compelling argument the mean-field approach. We demonstrate the fundamental difference between working with direct pr. capita optimization and using the mean field approach, explicitly quantifiying the difference in expected payoff for the case of the ideal free distribution. Behavior based on individual optimization appears more cautios than predicted when optimizing pr. capita consumption, which we demonstrate with an example and conjecture as a general property. We demonstrate our approach by applying it to a behaviorally modified Rosenzweig-MacArthur system in continuous space. We show that the system satisfies the criteria for existence and unique as a population game.


In addition to our theoretical advances, we implement a simple and efficient numerical method of finding Nash equilibria and equilbria of population games. The approach is applied to our case of the behaviorally modified Rosenzweig-MacArthur system. We examine the population dynamics through a phase portrait, where they appear to be asymptotically stable. We study the population levels and spatial distribution at equilibrium as a function of the carrying capacity, intraspecific predator competition and refuge quality.

%Add references to krivan and cressmans other articles, including the empirical one by krivan.

%The study of population games through the ideal free distribution has had great success in the ecological litterature, but it has the weakness of being unable to handle games with non-linear interactions. Here we solve this problem by introducing results from the study of mean field games in mathematical optimization.


%In the study of habitat-choice games, the ideal free distribution is the correct choice, in contrast to full population optimization (cite monroe), with exceptions where inherently monomorphic populations are a resaonable assumption.
%The theory of population games has developed in the area of mathematical optimization, \citep{aumann} and mathematical biology simultaneously \citep{cressman,kvivran,etc}. In the field of mathematical optimization, it is typically described as static mean field game theory and is characterized by arriving at the ideal free distribution mathematical biology.
