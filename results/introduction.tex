\section{Introduction}
Game theory is a natural tool to model the behavior of animals, who must respond to the behavior of other animals as well as complex and rapidly shifting environments. A classical application of game-theory is patch-choice models, where the ideal free distribution emerges to explain spatial distributions of populations \citep{cressman2004ideal}. A game theoretical approach has been fruitful in studying habitat choice in simple ecosystems under the assumption of static populations or simplifying the habitat to a few discrete patches, \citep{cressman2010ideal, valdovinos2010consequences}. Real-life habitat choice consists of animals choosing where to forage in a continuous landscape, with varying intra-specific competition and external risk factors. Models that can handle such systems allow for better models of habitat distribution, and represent a significant step forward in understanding natural systems \citep{morris2003shadows}.

A common simplification when including behavior in population models by using game-theory is to assume that at least one payoff is linear in the choice of strategy, \citep{krivan1997dynamic}. Linear models are suffficient to explain simple predator-prey dynamics with optimal behavior, \citep{kvrivan2007lotka}, but non-linear effects in natural systems are substantial \citep{gross2009generalized}. A population game is a system of interacting populations where each individual chooses the best strategy at every instant. That is, population games generalize the ideal free distribution \citep{cressman2004ideal} when each individual maximizes their fitness. A general model for population games is set out in \citep{vincent2005evolutionary} where optimal behavior is introduced by every population maximizing the per capita growth at every instant.  This implicitly assumes monomorphic populations, where all individuals intrinsically act as one \citep{malone2020ecology,stump2017optimally}. This assumption of monomorphic populations is the typical approach to population games with instantaneous migrations  \citep{kvrivan2013behavioral, vincent2005evolutionary}, but it is well-known that this does not generalize the ideal-free distribution and dramatically increases the per capita gain \citep{kvrivan2008ideal}. We propose a modification of the approach from \citep{vincent2005evolutionary} in the vein of \citep{cressman2010ideal}, based on individual optimization in the context of habitat selection. Instead of assuming a population where all individuals act in lockstep, we assume that each animal acts independently, but that its risk-reward calculus is affected by the population mean behavior, as in the ideal free distribution \citep{fretwell1969territorial}. This marks a return to the ideas of playing the field, \citep{smith1982evolution} and habitat selection games {cressman2010ideal}. If we assume large populations and that the animals of each type are indistinguishable, the game at every instant game can be modeled as a mean field game with multiple types.


The idea of instantanous movement where each individual maximizes its fitness based on full knowledge of the environment is the basic assumption of the ideal free distribution \citep{kvrivan2008ideal}. This assumption of instantaneous movement is, of course, only valid when the time-scales of movement and population dynamics are separated and the environment is interconnected \citep{cressman2006migration}. Though we model instantaneous movement, actual animals of course do not instantaneously but migrate between adjacent patches through advection-diffusion dynamics \citep{cantrell2010evolution}. When the population dynamics are on the same time-scale as the population dynamics this has a major impact \citep{abrams2007role}. A wide variety of migration dynamics in both continuous and discrete habitats give rise to the ideal free distribution \citep{averill2012several}, and generally the migration dynamics which lead to the simple ideal free distribution are those which are evolutionarily stable \citep{cantrell2010evolution}. The exception to this is in environments where diffusion fluxes plays a major role in the migration rather than active movement, in this case the evolutionarily stable strategies do not necessarily lead to exact ideal free distributions \citep{cantrell2010evolution}. The ideal free distribution as a game theoretic construction assumes that every individual has perfect information, but this is not a necessary requirement for its emergence at equilibrium \citep{flaxman2011evolutionary}. As such, populations at a population-dynamical equilibrium can be expected to follow a distribution where each individual has optimal fitness \citep{stephen2007ideal, cressman2010ideal}, which in this case is zero. If population dynamics are sufficiently slow, then the migration dynamics which lead to the ideal free distribution are evolutionarily stable \citep{cantrell2020evolution}, and even very simple migration dynamics lead to the ideal free distribution \citep{avgar2020habitat}. If this decoupling is not present, then the populations cannot be expected to follow the ideal free distribution at transient states \citep{abrams2007role, lou2014approaching}. As such, the model of instantaneous movement must be used with care, but is particularly suited for studying populations at steady-state \citep{cantrell2020evolution, cantrell2010evolution, cantrell2012evolutionaryb, cantrell2012evolutionary}. A general framework for adressing such system has not previously been given, we develop a method based on mean-field games to handle this.

This has not been done previously, as the evolution of mean-field games has followed two parallel tracks, one in mathematical biology through the ideal free distribution and habitat selection games \citep{fretwell1969territorial, parker1978searching, cressman2004ideal, kvrivan2008ideal, cressman2010ideal, broom2013game}, and the other in mathematical optimization based directly on anonymous actors \citep{lasry2007mean, aumann1964markets, blanchet2016optimal}. The main focus in the game-theoretically focused ecological work has been studying specific families of games in depth \cite{broom2013game}, while the focus in mathematical optimization has been in establishing uniqueness and existence of Nash equilibria through the toolbox of variational inequalities \citep{karamardian1969nonlinear, gabay1980uniqueness, nabetani2011parametrized}.



Using the theory of variational inequalities, we show that population games based on individual optimization have a unique equilibrium under very mild assumptions. Our approach allows us to handle both continuous and discrete strategy spaces, but more technical assumptions are required for existence in the continuous setting. The classical ideal free distribution emerges as a special case of our approach, providing a compelling argument for the mean-field approach. By working with variational inequalities, we can generalize the classical definition of a multi-species evolutionary stable state to the continuous setting \citep{cressman2001evolutionary}. We demonstrate our approach by applying it to a Rosenzweig-MacArthur system with intraspecific predator competition in continuous space. We modify the system so both predators and consumers have instantaneous optimal behavior based on maximizing the individual growth rate. We show that the Rosenzweig-MacArthur system with optimal behavior satisfies the criteria for existence and uniqueness of equilibria as a population game, and analyze the sensitivity of the system.


In addition to our theoretical advances, we implement a simple and efficient numerical method of finding Nash equilibria and equilbria of population games. The approach is applied to our case of the behaviorally modified Rosenzweig-MacArthur system. We examine the population dynamics through a phase portrait, where they appear to be asymptotically stable. We study the population levels and spatial distribution at equilibrium as a function of the carrying capacity, intraspecific predator competition. The rest of the paper is organized as follows: We start with the general setting. After building the general setting, we introduce the machinery of variational inequalities in the context of game theory. Here we show the general uniqueness and existence results. We proceed to study the concrete Rosenzweig-MacArthur model, showing existence and uniqueness of the Nash equilibrium and population equilibrium. We analyze the results, and discuss the implications of both the results and our theoretical results.
